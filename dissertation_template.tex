\documentclass[12pt,oneside]{book}

% packages
\usepackage{amssymb, amsmath, amsthm}
\usepackage{graphicx}
\usepackage{setspace}
\usepackage{enumerate}
\usepackage{bm}
\usepackage{subcaption}
\usepackage{caption}
\usepackage{amssymb}
\usepackage{cases}
\usepackage{booktabs}
\usepackage{algorithm}
\usepackage{algorithmic}
% these make top, right, bottom margins about 1"
% and the left margin about 1.5"
\setlength{\textwidth}{5.75in}
\setlength{\oddsidemargin}{0.5in}
\setlength{\evensidemargin}{0.0in}
\setlength{\textheight}{9.0in}
\setlength{\topmargin}{0.0in}
%\usepackage[top=1in, bottom=1in, left=1.5 in, right=1 in]{geometry}

% this makes the chapters start with a 2" margin
\makeatletter
\renewcommand*\@makechapterhead[1]{%
  \vspace*{0.75in}%
  {\parindent \z@ \raggedright \normalfont
    \ifnum \c@secnumdepth >\m@ne
        \huge\bfseries \@chapapp\space \thechapter
        \par\nobreak
        \vskip 20\p@
    \fi
    \interlinepenalty\@M
    \Huge \bfseries #1\par\nobreak
    \vskip 40\p@
  }}
\renewcommand*\@makeschapterhead[1]{%
  \vspace*{0.75in}%
  {\parindent \z@ \raggedright
    \normalfont
    \interlinepenalty\@M
    \Huge \bfseries  #1\par\nobreak
    \vskip 40\p@
  }}
\makeatother





% The next lines are personal shortcuts and settings.  Feel free
% to delete those you don't want.

%  theorems, corollaries, etc.
\theoremstyle{plain}
\newtheorem{thm}{Theorem}[chapter]
\newtheorem{cor}[thm]{Corollary}
\newtheorem{lem}[thm]{Lemma}
\newtheorem{prop}[thm]{Proposition}

\theoremstyle{definition}
\newtheorem{example}[thm]{Example}
\newtheorem{dfn}[thm]{Definition}
\newtheorem{exercise}[thm]{Exercise}

% QED symbol is a filled box, instead of unfilled box
\renewcommand{\qedsymbol}{\vrule height 5pt width 4pt depth -1pt}

% text shortcuts
\def\ie{{i.e.,}\ }
\def\eg{{e.g.,}\ }
\def\bbar#1{\overline{#1}}
\def\bigskip{\vspace{30pt}}

% math symbol shortcuts
\newcommand{\nor}{\trianglelefteq} % \nor = normal subgroup symbol
\newcommand{\snor}{\triangleleft} % \snor = proper normal subgroup
\newcommand{\cross}{^{\times}} % \cross = direct product
\newcommand{\iso}{\cong} % \iso = isomorphic symbol
\newcommand{\divides}{\bigm |} % \divides = divides symbol
\def\ord#1{\vert #1 \vert} % \ord G = |G|
\def\ind#1#2{\vert #1\, :\,#2\vert}
\def\gcd#1#2{\textrm{gcd}(#1, #2)}
\def\deg{\textrm{deg}}

%  group shortcuts
% \Aut {G} = Aut G, etc.
\def\Aut#1{\textrm{Aut } #1}
\def\Out#1{\textrm{Out } #1}
\def\Inn#1{\textrm{Inn } #1}
\def\Hom{\textrm{Hom}}
\def\Fit#1{\textrm{Fit } #1}
\def\Z#1{\textrm{Z}(#1)}

\def\syl#1{Syl_{#1} (G)}
\def\sltwo#1{\textrm{SL}_2(#1)}
\def\gltwo#1{\textrm{GL}_2(#1)}
\def\gl#1#2{\textrm{GL}_{#1}(#2)}
\def\psl#1#2{\textrm{PSL}_{#1}(#2)}

%  set shortcuts: blackboard bold commands
\newcommand{\QQ}{\mathbb{Q}}
\newcommand{\ZZ}{\mathbb{Z}}
\newcommand{\RR}{\mathbb{R}}
\newcommand{\CC}{\mathbb{C}}
\newcommand{\NN}{\mathbb{N}}
\newcommand{\FF}{\mathbb{F}}
\newcommand{\OO}{\mathbb{O}}

% logic shortcuts
\renewcommand{\iff}{\Leftrightarrow}

\newcommand{\newheading}[1]
{ \begin{center} {\large \bf #1} \end{center} \vspace{0.5in} }
\newcommand{\newchapter}[2] { \newpage \chapter{#1} \label{chap:#2} \input{#2} }
\newcommand{\tab}{\phantom{o}\hspace{2ex}}
\newcommand{\supercite}[1]{\textsuperscript{\cite{#1}}}

% doesn't count introduction when numbering theorems, etc.
\setcounter{chapter}{0}

% makes page numbers appear
\pagestyle{plain}


\begin{document}

% makes the page numbers roman numerals, doesn't count
% these pages in the table of contents
\frontmatter

\thispagestyle{empty}

\vbox to 1truein{}

\centerline{GRAPH PARTITIONING}

\vskip 150pt

\centerline{BY}
\vskip 10pt

\centerline{ZHICHENG ZHU}
\vskip 10pt

\centerline{B.S., Some School, Year}
\centerline{M.S., Some School, Year}

\vskip 200pt

\centerline{THESIS}
\vskip 10pt

\centerline{Submitted in partial fulfillment of the requirements for}
\centerline{the degree of Master of Science in Computer Science}
\centerline{in the Graduate School of}
\centerline{Binghamton University}
\centerline{State University of New York}
\centerline{2013}

\newpage

\thispagestyle{empty}

\vbox to 8.0truein{}

\centerline{\copyright\ Copyright by Zhicheng Zhu, 2013}

\

\centerline{All Rights Reserved}

\newpage

{\baselineskip = 10pt

\vbox to 2.0truein{}


\centerline{Accepted in partial fulfillment of the requirements for}
\centerline{the degree of Master of Science in Computer Science}
\centerline{in the Graduate School of}
\centerline{Binghamton University}
\centerline{State University of New York}
\centerline{Year}
\vskip 60pt

\centerline{December 2nd, 2013}
\vskip 70pt

\centerline{Patrick H. Madden, Chair and Faculty Advisor}
\centerline{Department of Computer Science, Binghamton University}
\vskip 50pt

\centerline{Timothy Normand Miller, Member}
\centerline{Department of Computer Science, Binghamton University}
\vskip 50pt

% \centerline{Committee Person 2, Member}
% \centerline{Department of Mathematical Sciences, Binghamton University}
% \vskip 50pt

% \centerline{Committee Person 3, Outside Examiner}
% \centerline{Location of Outside Examiner}


}

\newpage

\cite{remove-me}


\newpage

\tableofcontents
\listoffigures
\listoftables

% Changes page numbers to regular numbers, resets the counter
\mainmatter

% This gives 11pt font with 20pt spacing, text from here should be double spaced
\fontsize{11}{20pt} \selectfont

% \include puts in the .tex file with the given name
% make sure that these files don't have any preamble material
\newchapter{Introduction}{introduction}
% this chapter has sample text for the bibliography
% just delete it and replace with your own
\newchapter{Graph Partition}{chapter1}
\newchapter{Hmetis}{chapter2}
\newchapter{Multigrid-based algorithm for circuit clustering}{chapter3}
\newchapter{Two ways to combine a new cluster algorithm and hemtis}{chapter4}
\newchapter{Experiments and analysis}{chapter5}
\newchapter{Conclusions and Future Work}{chapter6}


% add a new chapter without a chapter # for the references
\addcontentsline{toc}{chapter}{Bibliography}

% bibs need to be single-spaced

\fontsize{11}{11pt} \selectfont

% specifies the style for the bibliography and inputs the
% bibtex file references.bib for the bibliography
\bibliographystyle{plain}
\bibliography{references}


\end{document}
